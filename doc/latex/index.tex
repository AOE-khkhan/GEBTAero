G\+E\+B\+T\+Aero is an aeroelasticity simulation toolbox with a computation code coded in Fortran and a pre/postprocessor coded in Python. The computation code is derived from G\+E\+BT program developped by Prof. Yu (\href{https://cdmhub.org/resources/gebt}{\tt https\+://cdmhub.\+org/resources/gebt}). The pre/postprocessor uses several open source programs available in most linux distros repositories\+:
\begin{DoxyItemize}
\item calculix \+: a Finite Element Method solver (\href{http://www.calculix.de/}{\tt http\+://www.\+calculix.\+de/})
\item paraview \+: a data analysis and visualization application (\href{https://www.paraview.org/}{\tt https\+://www.\+paraview.\+org/})
\item Mumps \+: a parallel sparse direct solver (\href{http://mumps.enseeiht.fr/}{\tt http\+://mumps.\+enseeiht.\+fr/})
\item Arpack \+: a sparse eigenvalue solver (\href{https://www.caam.rice.edu/software/ARPACK/}{\tt https\+://www.\+caam.\+rice.\+edu/software/\+A\+R\+P\+A\+C\+K/})
\end{DoxyItemize}

\subsection*{Installation}

Two options are available\+:

\subsubsection*{Debian package}

For Ubuntu 18.\+04 and Debian 10, download the .deb file available in the package folder and launch it. It will automatically install all the dependancies. For other linux distributions, you can ask for a .deb or .rpm package creation.

\subsubsection*{Compilation}

Clone the repository use the Make\+File in the bin folder and adapt it to your system.

Install the dependancies. On Ubuntu\+: 
\begin{DoxyCode}
sudo apt install paraview calculix-ccx calculix-cgx libmumps-seq-dev libarpack2-dev python3 python3-numpy
       python3-matplotlib gfortran make
\end{DoxyCode}
 Compile gebtaero and unical (mesh format translator from unv to inp)

\subsection*{Testing}

The folder cas\+\_\+test is a set of automated python script designed to test many program functionalities. in a terminal launch\+: 
\begin{DoxyCode}
python3 tests.py
\end{DoxyCode}


\subsection*{Usage}

Besides cas\+\_\+test folder, examples folder contains a set of detailled script designed to help you to set your own problems. The pre/postprocessor script must be launch with python3 (not python2). 
\begin{DoxyCode}
python3 myscript.py
\end{DoxyCode}
 You can also directly use the computation code with .dat file (show examples)\+: 
\begin{DoxyCode}
gebtaero example.dat
\end{DoxyCode}
 It will generate a .ech text file summarising the input parameters, a .out text file with the output data and optionally a vtk file folder

You can use an I\+DE to easily launch the scripts and read the source code. This program has been developed using geany (www.\+geany.\+org). Open a python script and press \char`\"{}\+Maj+\+F5\char`\"{}. Modify \char`\"{}python\char`\"{} in \char`\"{}python3\char`\"{} and press \char`\"{}\+F5\char`\"{} to launch the script.

\subsection*{Documentation}

If you have installed the debian package, an application \char`\"{}\+G\+E\+B\+T\+Aero Doc\char`\"{} is installed on your system. It launches an html page with the whole documentation of the sources (Fortran and Python) You can also generate the documentation using doxygen and the file \char`\"{}\+Doxyfile\char`\"{} in the \char`\"{}doc\char`\"{} folder.

\subsection*{License}

See the License file in the repository

\subsection*{Acknowledgement}

This research work was funded by the French Air Force Academy Research Center in collaboration with I\+S\+A\+E-\/\+Supaero 